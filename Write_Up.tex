\documentclass[11pt,preprint, authoryear]{elsarticle}

\usepackage{lmodern}
%%%% My spacing
\usepackage{setspace}
\setstretch{1.2}
\DeclareMathSizes{12}{14}{10}{10}

% Wrap around which gives all figures included the [H] command, or places it "here". This can be tedious to code in Rmarkdown.
\usepackage{float}
\let\origfigure\figure
\let\endorigfigure\endfigure
\renewenvironment{figure}[1][2] {
    \expandafter\origfigure\expandafter[H]
} {
    \endorigfigure
}

\let\origtable\table
\let\endorigtable\endtable
\renewenvironment{table}[1][2] {
    \expandafter\origtable\expandafter[H]
} {
    \endorigtable
}


\usepackage{ifxetex,ifluatex}
\usepackage{fixltx2e} % provides \textsubscript
\ifnum 0\ifxetex 1\fi\ifluatex 1\fi=0 % if pdftex
  \usepackage[T1]{fontenc}
  \usepackage[utf8]{inputenc}
\else % if luatex or xelatex
  \ifxetex
    \usepackage{mathspec}
    \usepackage{xltxtra,xunicode}
  \else
    \usepackage{fontspec}
  \fi
  \defaultfontfeatures{Mapping=tex-text,Scale=MatchLowercase}
  \newcommand{\euro}{€}
\fi

\usepackage{amssymb, amsmath, amsthm, amsfonts}

\def\bibsection{\section*{References}} %%% Make "References" appear before bibliography


\usepackage[round]{natbib}

\usepackage{longtable}
\usepackage[margin=2.3cm,bottom=2cm,top=2.5cm, includefoot]{geometry}
\usepackage{fancyhdr}
\usepackage[bottom, hang, flushmargin]{footmisc}
\usepackage{graphicx}
\numberwithin{equation}{section}
\numberwithin{figure}{section}
\numberwithin{table}{section}
\setlength{\parindent}{0cm}
\setlength{\parskip}{1.3ex plus 0.5ex minus 0.3ex}
\usepackage{textcomp}
\renewcommand{\headrulewidth}{0.2pt}
\renewcommand{\footrulewidth}{0.3pt}

\usepackage{array}
\newcolumntype{x}[1]{>{\centering\arraybackslash\hspace{0pt}}p{#1}}

%%%%  Remove the "preprint submitted to" part. Don't worry about this either, it just looks better without it:
\makeatletter
\def\ps@pprintTitle{%
  \let\@oddhead\@empty
  \let\@evenhead\@empty
  \let\@oddfoot\@empty
  \let\@evenfoot\@oddfoot
}
\makeatother

 \def\tightlist{} % This allows for subbullets!

\usepackage{hyperref}
\hypersetup{breaklinks=true,
            bookmarks=true,
            colorlinks=true,
            citecolor=blue,
            urlcolor=blue,
            linkcolor=blue,
            pdfborder={0 0 0}}


% The following packages allow huxtable to work:
\usepackage{siunitx}
\usepackage{multirow}
\usepackage{hhline}
\usepackage{calc}
\usepackage{tabularx}
\usepackage{booktabs}
\usepackage{caption}


\newenvironment{columns}[1][]{}{}

\newenvironment{column}[1]{\begin{minipage}{#1}\ignorespaces}{%
\end{minipage}
\ifhmode\unskip\fi
\aftergroup\useignorespacesandallpars}

\def\useignorespacesandallpars#1\ignorespaces\fi{%
#1\fi\ignorespacesandallpars}

\makeatletter
\def\ignorespacesandallpars{%
  \@ifnextchar\par
    {\expandafter\ignorespacesandallpars\@gobble}%
    {}%
}
\makeatother

\newlength{\cslhangindent}
\setlength{\cslhangindent}{1.5em}
\newenvironment{CSLReferences}%
  {\setlength{\parindent}{0pt}%
  \everypar{\setlength{\hangindent}{\cslhangindent}}\ignorespaces}%
  {\par}


\urlstyle{same}  % don't use monospace font for urls
\setlength{\parindent}{0pt}
\setlength{\parskip}{6pt plus 2pt minus 1pt}
\setlength{\emergencystretch}{3em}  % prevent overfull lines
\setcounter{secnumdepth}{5}

%%% Use protect on footnotes to avoid problems with footnotes in titles
\let\rmarkdownfootnote\footnote%
\def\footnote{\protect\rmarkdownfootnote}
\IfFileExists{upquote.sty}{\usepackage{upquote}}{}

%%% Include extra packages specified by user
\usepackage{array}
\usepackage{caption}
\usepackage{graphicx}
\usepackage{siunitx}
\usepackage[normalem]{ulem}
\usepackage{colortbl}
\usepackage{multirow}
\usepackage{hhline}
\usepackage{calc}
\usepackage{tabularx}
\usepackage{threeparttable}
\usepackage{wrapfig}
\usepackage{adjustbox}
\usepackage{hyperref}

%%% Hard setting column skips for reports - this ensures greater consistency and control over the length settings in the document.
%% page layout
%% paragraphs
\setlength{\baselineskip}{12pt plus 0pt minus 0pt}
\setlength{\parskip}{12pt plus 0pt minus 0pt}
\setlength{\parindent}{0pt plus 0pt minus 0pt}
%% floats
\setlength{\floatsep}{12pt plus 0 pt minus 0pt}
\setlength{\textfloatsep}{20pt plus 0pt minus 0pt}
\setlength{\intextsep}{14pt plus 0pt minus 0pt}
\setlength{\dbltextfloatsep}{20pt plus 0pt minus 0pt}
\setlength{\dblfloatsep}{14pt plus 0pt minus 0pt}
%% maths
\setlength{\abovedisplayskip}{12pt plus 0pt minus 0pt}
\setlength{\belowdisplayskip}{12pt plus 0pt minus 0pt}
%% lists
\setlength{\topsep}{10pt plus 0pt minus 0pt}
\setlength{\partopsep}{3pt plus 0pt minus 0pt}
\setlength{\itemsep}{5pt plus 0pt minus 0pt}
\setlength{\labelsep}{8mm plus 0mm minus 0mm}
\setlength{\parsep}{\the\parskip}
\setlength{\listparindent}{\the\parindent}
%% verbatim
\setlength{\fboxsep}{5pt plus 0pt minus 0pt}



\begin{document}



%titlepage
\thispagestyle{empty}
\begin{center}
\begin{minipage}{0.75\linewidth}
    \centering
%Entry1
    {\uppercase{\huge COVID-19 vaccine Lottery Field Experiment\par}}
    \vspace{2cm}
%Author's name
    {\LARGE \textbf{Cassandra Pengelly}\par}
    \vspace{1cm}
%University logo
%Supervisor's Details
\begin{center}
    {\Large Behavioural Economics 871 Essay\par}
    \vspace{1cm}
%Degree
    {\large 15 October 2021\par}
    \vspace{1cm}
%Institution
    {\large \par}
    \vspace{1cm}
%Date
    {\large }
%More
    {\normalsize }
%More
    {\normalsize }
\end{center}
\end{minipage}
\end{center}
\clearpage


\begin{frontmatter}  %

\title{}

% Set to FALSE if wanting to remove title (for submission)


\vspace{1cm}





\vspace{0.5cm}

\end{frontmatter}


\renewcommand{\contentsname}{Table of Contents}
{\tableofcontents}

%________________________
% Header and Footers
%%%%%%%%%%%%%%%%%%%%%%%%%%%%%%%%%
\pagestyle{fancy}
\chead{}
\rhead{}
\lfoot{}
\rfoot{\footnotesize Page \thepage}
\lhead{}
%\rfoot{\footnotesize Page \thepage } % "e.g. Page 2"
\cfoot{}

%\setlength\headheight{30pt}
%%%%%%%%%%%%%%%%%%%%%%%%%%%%%%%%%
%________________________

\headsep 35pt % So that header does not go over title




\newpage

\hypertarget{introduction}{%
\section{\texorpdfstring{Introduction
\label{Introduction}}{Introduction }}\label{introduction}}

One critical mechanism through which epidemics are contained is through
herd immunity. Herd immunity is where a certain amount of the population
is immune to a disease and can be achieved either through a certain
threshold of the population being vaccinated or being exposed directly
to the disease. Ethically, herd immunity should be achieved through
vaccination. In spite of this, many South Africans are hesitant to get
the COVID vaccine; \protect\hyperlink{ref-stat}{RSA}
(\protect\hyperlink{ref-stat}{2021}) reports that only 25\% of South
Africa's adult population have been fully vaccinated\footnote{Statistics
  reported as of 11 October 2021}. Governments around the world are
devising strategies to contain the pandemic; one solution is to
incentivise vaccinations. Here, insights from behaviourial economics
could provide a cost-effective solution: a vaccine lottery.

This essay proposes a field experiment to investigate whether a vaccine
lottery could improve vaccination rates in South Africa. Section
\ref{lit} briefly reviews the relevant literature on behavioural
economics and vaccine lotteries. Section \ref{design} describes the
design of the experiment and outlines the three types of treatment
groups. Section \label{treat} discusses how the treatment will be
administered and how the data will be collected. Lastly, section
\ref{pre} gives a pre-analysis plan of the empirical analysis that will
be performed on the data; and the final section (\ref{con}) concludes.

• a clear statement of the research question and motivation for why this
is interesting and important; • a brief review of the relevant
literature (both theoretical and empirical) which highlights the
research gap your experiment will address;

• a clear description of the experimental design and the theory of
change; • an explanation of how the treatments will be administered and
data gathered (including proposed partner institutions); • a
pre-analysis plan of the empirical analysis that will be performed on
the data.

amount of people have received the vaccine so far. Critical to reach
herd immunity is improving vaccination rates.

\hypertarget{experiment-design}{%
\section{\texorpdfstring{Experiment Design
\label{design}}{Experiment Design }}\label{experiment-design}}

The field experiment is designed Research question: could a vaccine
lottery improve vaccination rates in South Africa?

Have 1 control group: no messages, no lotto Have 4 treatment groups: •
Send messages • 1 lottery where you are entered if you got at least 1
vaccination • 1 lottery where a friend has to refer you • 1 regret
lottery: everyone is entered into the lotto (receive sms) and can only
win if vaccinated Can randomize across municipalities or provinces size
of sample population for experiment

Many South Africans are hesitant to get the COVID vaccine. As STATSA
shows, amount of people have received the vaccine so far. Critical to
reach herd immunity is improving vaccination rates. In order to improve
the take-up of vaccinations, a field experiment designed around a
vaccination lottery is proposed. While some governments have considered
and experimented with lump-sum payments, behavioural economics could
provide a more cost-effective solution. Individuals have a tendency to
overweight small probabilities and this overestimate their chances of
winning a lottery. In South Africa, Lit Review Theory: overweight small
probabilities, gambling, social preferences, regret avoidance Empirical:
vaccine field designs, lottery incentives, regret lottery incentives
Several authors have experimented with lotteries as an incentive for
vaccinations, although there have been no studies as of yet on the South
African population. Probles with vaccine studies: too few participants.
A larger study could fill this gap in the literature.

\hypertarget{treatment-and-data}{%
\section{\texorpdfstring{Treatment and Data
\label{treat}}{Treatment and Data }}\label{treatment-and-data}}

\hypertarget{partner-institutions-and-funding}{%
\section{\texorpdfstring{Partner Institutions and Funding
\label{part}}{Partner Institutions and Funding }}\label{partner-institutions-and-funding}}

\hypertarget{pre-analysis-plan}{%
\section{\texorpdfstring{Pre-analysis plan
\label{pre}}{Pre-analysis plan }}\label{pre-analysis-plan}}

Students must submit an essay in which they design a field experiment
that could answer an interesting behavioural economic question. The
essay must contain the following:

• a clear statement of the research question and motivation for why this
is interesting and important; • a brief review of the relevant
literature (both theoretical and empirical) which highlights the
research gap your experiment will address;

• a clear description of the experimental design and the theory of
change; • an explanation of how the treatments will be administered and
data gathered (including proposed partner institutions); • a
pre-analysis plan of the empirical analysis that will be performed on
the data.

Overview In this field experiment, a person who refers his/her friend to
receive a vaccine would be entered into a lucky draw, with a monetary
prize, created by the government. The purpose behind this nudge is to
encourage people who would otherwise not have got a Covid vaccine, to do
so. The hypothesis is that there should be an increase in the total
number of people receiving a Covid vaccine after the nudge is
implemented. Increasing the number of vaccinations is important as
medical research shows that vaccines decrease the probability of
contracting Covid-19 and are also effective at reducing the severity of
the symptoms of the virus for those who do contract it. The Nudge The
nudge addresses behaviour by creating an environment where there is
social pressure to get a vaccine (if I wanted to enter the lucky draw, I
would pressure my friend into getting the vaccine). It is also likely
that if a person asks her friend to get the vaccine so she can enter the
lucky draw, she will reciprocate and get the vaccine as well so that her
friend may enter the draw, which will also increase the number of people
getting vaccinated. For the vaccines that require two doses
(e.g.~Pfizer), a person's name could be withdrawn, if the second shot is
not given within a certain amount of time. This makes use of loss
aversion, where people who already have their names in the draw feel the
pain of having their names withdrawn more intensely than the pleasure of
having their names added a second time to the draw for getting their
second shot. Target Group The lucky draw is anticipated to attract
people who are risk-on (they enjoy gambling, and are less worried about
getting vaccinated), and poorer individuals for whom winning money is
more attractive. These target groups are desirable as they are less
likely to get the vaccine, and the government would like to maximise the
number of vaccinated people. Additionally, if there are individuals who
want to be vaccinated but procrastinate getting the vaccine (e.g.~naïve
hyperbolic discounters), setting a deadline for the lucky draw could
increase the utility of getting the vaccine earlier enough to overcome
the procrastination problem. There is no downside or extra cost for
having people enter the lucky draw who would otherwise still have got
the vaccine. Proposed Partner Institutions This field experiment would
be in collaboration with the South African government and facilities
that conduct vaccinations (e.g.~Clicks). The government would be where
the data is centralized and the administers of vaccines would all be
data collection nodes. After a person has received a vaccine, the
administer would ask if the person received a referral for the shot, and
then note the ID number of the friend in addition to the individual's
details.

Data Collection There is a data collection system already set up at the
vaccination sites so this extra data point would not be difficult to
collect within the current tracking system. Depending on costs, the
referral friend could be sent an sms thanking her for caring about
others and getting them vaccinated, and letting her know that she has
been entered into the draw. This is a positive reinforcement technique
and shows people that the government is following up on their promise.
This acknowledgement and transparency is expected to encourage more
referrals. Once the lucky draw has been concluded, the data can be
analysed, the purpose of which is to uncover whether the nudge increased
vaccinations.

References are to be made as follows:
\protect\hyperlink{ref-fama1997}{Fama \& French}
(\protect\hyperlink{ref-fama1997}{1997: 33}) and
\protect\hyperlink{ref-grinold2000}{Grinold \& Kahn}
(\protect\hyperlink{ref-grinold2000}{2000}) Such authors could also be
referenced in brackets (\protect\hyperlink{ref-grinold2000}{Grinold \&
Kahn, 2000}) and together \protect\hyperlink{ref-grinold2000}{Grinold \&
Kahn} (\protect\hyperlink{ref-grinold2000}{2000}). Source the reference
code from scholar.google.com by clicking on ``cite'\,' below article
name. Then select BibTeX at the bottom of the Cite window, and proceed
to copy and paste this code into your ref.bib file, located in the
directory's Tex folder. Open this file in Rstudio for ease of
management, else open it in your preferred Tex environment. Add and
manage your article details here for simplicity - once saved, it will
self-adjust in your paper.

\begin{quote}
I suggest renaming the top line after @article, as done in the template
ref.bib file, to something more intuitive for you to remember. Do not
change the rest of the code. Also, be mindful of the fact that bib
references from google scholar may at times be incorrect. Reference
Latex forums for correct bibtex notation.
\end{quote}

\begin{figure}[H]

{\centering \includegraphics{Write_Up_files/figure-latex/Figure1-1} 

}

\caption{Caption Here \label{Figure1}}\label{fig:Figure1}
\end{figure}

According to \protect\hyperlink{ref-stat}{RSA}
(\protect\hyperlink{ref-stat}{2021}), 34\% of South Africans are
vaccinated, while only 25\% are fully vaccinated.\\

\begin{table}[H]
\centering
\begin{tabular}{llll}
  \toprule
Province & Total Adults Vaccinated & Adult Population & Percentage Vaccinated \\ 
  \midrule
Eastern Cape & 1 603 045 & 4 099 543 & 39\% \\ 
  Free State & 735 696 & 1 914 521 & 38\% \\ 
  Gauteng & 3 523 373 & 11 311 326 & 31\% \\ 
  KwaZulu-Natal & 2 170 526 & 7 219 795 & 30\% \\ 
  Limpopo & 1 437 846 & 3 695 801 & 39\% \\ 
  Mpumalanga & 831 759 & 3 039 520 & 27\% \\ 
  North West & 835 206 & 2 693 247 & 31\% \\ 
  Northern Cape & 290 962 & 847 545 & 34\% \\ 
  Western Cape & 2 141 933 & 4 976 903 & 43\% \\ 
  Total & 13 570 346 & 39 798 201 & 34\% \\ 
   \bottomrule
\end{tabular}
\caption{Vaccination Statistics \label{tab1}} 
\end{table}

To reference calculations \textbf{in text}, \emph{do this:} From table
\ref{tab1} we see the average value of mpg is 20.98.

\begingroup\fontsize{12pt}{13pt}\selectfont
\begin{longtable}{rrrrrrrrrrr}
\caption{Long Table Example} \\ 
  \toprule
mpg & cyl & disp & hp & drat & wt & qsec & vs & am & gear & carb \\ 
  \hline 
\endhead 
\hline 
{\footnotesize Continued on next page} 
\endfoot 
\endlastfoot 
 \midrule
21.00 & 6.00 & 160.00 & 110.00 & 3.90 & 2.62 & 16.46 & 0.00 & 1.00 & 4.00 & 4.00 \\ 
  21.00 & 6.00 & 160.00 & 110.00 & 3.90 & 2.88 & 17.02 & 0.00 & 1.00 & 4.00 & 4.00 \\ 
  22.80 & 4.00 & 108.00 & 93.00 & 3.85 & 2.32 & 18.61 & 1.00 & 1.00 & 4.00 & 1.00 \\ 
  21.40 & 6.00 & 258.00 & 110.00 & 3.08 & 3.21 & 19.44 & 1.00 & 0.00 & 3.00 & 1.00 \\ 
  18.70 & 8.00 & 360.00 & 175.00 & 3.15 & 3.44 & 17.02 & 0.00 & 0.00 & 3.00 & 2.00 \\ 
  18.10 & 6.00 & 225.00 & 105.00 & 2.76 & 3.46 & 20.22 & 1.00 & 0.00 & 3.00 & 1.00 \\ 
  14.30 & 8.00 & 360.00 & 245.00 & 3.21 & 3.57 & 15.84 & 0.00 & 0.00 & 3.00 & 4.00 \\ 
  24.40 & 4.00 & 146.70 & 62.00 & 3.69 & 3.19 & 20.00 & 1.00 & 0.00 & 4.00 & 2.00 \\ 
  22.80 & 4.00 & 140.80 & 95.00 & 3.92 & 3.15 & 22.90 & 1.00 & 0.00 & 4.00 & 2.00 \\ 
  19.20 & 6.00 & 167.60 & 123.00 & 3.92 & 3.44 & 18.30 & 1.00 & 0.00 & 4.00 & 4.00 \\ 
  17.80 & 6.00 & 167.60 & 123.00 & 3.92 & 3.44 & 18.90 & 1.00 & 0.00 & 4.00 & 4.00 \\ 
  16.40 & 8.00 & 275.80 & 180.00 & 3.07 & 4.07 & 17.40 & 0.00 & 0.00 & 3.00 & 3.00 \\ 
  17.30 & 8.00 & 275.80 & 180.00 & 3.07 & 3.73 & 17.60 & 0.00 & 0.00 & 3.00 & 3.00 \\ 
  15.20 & 8.00 & 275.80 & 180.00 & 3.07 & 3.78 & 18.00 & 0.00 & 0.00 & 3.00 & 3.00 \\ 
  10.40 & 8.00 & 472.00 & 205.00 & 2.93 & 5.25 & 17.98 & 0.00 & 0.00 & 3.00 & 4.00 \\ 
  10.40 & 8.00 & 460.00 & 215.00 & 3.00 & 5.42 & 17.82 & 0.00 & 0.00 & 3.00 & 4.00 \\ 
  14.70 & 8.00 & 440.00 & 230.00 & 3.23 & 5.34 & 17.42 & 0.00 & 0.00 & 3.00 & 4.00 \\ 
  32.40 & 4.00 & 78.70 & 66.00 & 4.08 & 2.20 & 19.47 & 1.00 & 1.00 & 4.00 & 1.00 \\ 
  30.40 & 4.00 & 75.70 & 52.00 & 4.93 & 1.61 & 18.52 & 1.00 & 1.00 & 4.00 & 2.00 \\ 
  33.90 & 4.00 & 71.10 & 65.00 & 4.22 & 1.83 & 19.90 & 1.00 & 1.00 & 4.00 & 1.00 \\ 
  21.50 & 4.00 & 120.10 & 97.00 & 3.70 & 2.46 & 20.01 & 1.00 & 0.00 & 3.00 & 1.00 \\ 
  15.50 & 8.00 & 318.00 & 150.00 & 2.76 & 3.52 & 16.87 & 0.00 & 0.00 & 3.00 & 2.00 \\ 
  15.20 & 8.00 & 304.00 & 150.00 & 3.15 & 3.44 & 17.30 & 0.00 & 0.00 & 3.00 & 2.00 \\ 
  13.30 & 8.00 & 350.00 & 245.00 & 3.73 & 3.84 & 15.41 & 0.00 & 0.00 & 3.00 & 4.00 \\ 
  19.20 & 8.00 & 400.00 & 175.00 & 3.08 & 3.85 & 17.05 & 0.00 & 0.00 & 3.00 & 2.00 \\ 
  27.30 & 4.00 & 79.00 & 66.00 & 4.08 & 1.94 & 18.90 & 1.00 & 1.00 & 4.00 & 1.00 \\ 
  26.00 & 4.00 & 120.30 & 91.00 & 4.43 & 2.14 & 16.70 & 0.00 & 1.00 & 5.00 & 2.00 \\ 
  30.40 & 4.00 & 95.10 & 113.00 & 3.77 & 1.51 & 16.90 & 1.00 & 1.00 & 5.00 & 2.00 \\ 
  15.80 & 8.00 & 351.00 & 264.00 & 4.22 & 3.17 & 14.50 & 0.00 & 1.00 & 5.00 & 4.00 \\ 
  19.70 & 6.00 & 145.00 & 175.00 & 3.62 & 2.77 & 15.50 & 0.00 & 1.00 & 5.00 & 6.00 \\ 
  15.00 & 8.00 & 301.00 & 335.00 & 3.54 & 3.57 & 14.60 & 0.00 & 1.00 & 5.00 & 8.00 \\ 
  21.40 & 4.00 & 121.00 & 109.00 & 4.11 & 2.78 & 18.60 & 1.00 & 1.00 & 4.00 & 2.00 \\ 
   \bottomrule
\end{longtable}
\endgroup

\hfill

\hypertarget{huxtable}{%
\subsection{Huxtable}\label{huxtable}}

Huxtable is a very nice package for making working with tables between
Rmarkdown and Tex easier.

This cost some adjustment to the Tex templates to make it work, but it
now works nicely.

See documentation for this package
\href{https://hughjonesd.github.io/huxtable/huxtable.html}{here}. A
particularly nice addition of this package is for making the printing of
regression results a joy (see
\href{https://hughjonesd.github.io/huxtable/huxtable.html\#creating-a-regression-table}{here}).
Here follows an example:

 
  \providecommand{\huxb}[2]{\arrayrulecolor[RGB]{#1}\global\arrayrulewidth=#2pt}
  \providecommand{\huxvb}[2]{\color[RGB]{#1}\vrule width #2pt}
  \providecommand{\huxtpad}[1]{\rule{0pt}{#1}}
  \providecommand{\huxbpad}[1]{\rule[-#1]{0pt}{#1}}

\begin{table}[ht]
\begin{centerbox}
\begin{threeparttable}
\captionsetup{justification=centering,singlelinecheck=off}
\caption{Regression Output}
 \label{Reg01}
\setlength{\tabcolsep}{0pt}
\begin{tabular}{l l l l}


\hhline{>{\huxb{0, 0, 0}{0.8}}->{\huxb{0, 0, 0}{0.8}}->{\huxb{0, 0, 0}{0.8}}->{\huxb{0, 0, 0}{0.8}}-}
\arrayrulecolor{black}

\multicolumn{1}{!{\huxvb{0, 0, 0}{0}}c!{\huxvb{0, 0, 0}{0}}}{\huxtpad{6pt + 1em}\centering \hspace{6pt} {\fontsize{12pt}{14.4pt}\selectfont } \hspace{6pt}\huxbpad{6pt}} &
\multicolumn{1}{c!{\huxvb{0, 0, 0}{0}}}{\huxtpad{6pt + 1em}\centering \hspace{6pt} {\fontsize{12pt}{14.4pt}\selectfont Reg1} \hspace{6pt}\huxbpad{6pt}} &
\multicolumn{1}{c!{\huxvb{0, 0, 0}{0}}}{\huxtpad{6pt + 1em}\centering \hspace{6pt} {\fontsize{12pt}{14.4pt}\selectfont Reg2} \hspace{6pt}\huxbpad{6pt}} &
\multicolumn{1}{c!{\huxvb{0, 0, 0}{0}}}{\huxtpad{6pt + 1em}\centering \hspace{6pt} {\fontsize{12pt}{14.4pt}\selectfont Reg3} \hspace{6pt}\huxbpad{6pt}} \tabularnewline[-0.5pt]


\hhline{>{\huxb{255, 255, 255}{0.4}}->{\huxb{0, 0, 0}{0.4}}->{\huxb{0, 0, 0}{0.4}}->{\huxb{0, 0, 0}{0.4}}-}
\arrayrulecolor{black}

\multicolumn{1}{!{\huxvb{0, 0, 0}{0}}l!{\huxvb{0, 0, 0}{0}}}{\huxtpad{6pt + 1em}\raggedright \hspace{6pt} {\fontsize{12pt}{14.4pt}\selectfont (Intercept)} \hspace{6pt}\huxbpad{6pt}} &
\multicolumn{1}{r!{\huxvb{0, 0, 0}{0}}}{\huxtpad{6pt + 1em}\raggedleft \hspace{6pt} {\fontsize{12pt}{14.4pt}\selectfont -2256.361 ***} \hspace{6pt}\huxbpad{6pt}} &
\multicolumn{1}{r!{\huxvb{0, 0, 0}{0}}}{\huxtpad{6pt + 1em}\raggedleft \hspace{6pt} {\fontsize{12pt}{14.4pt}\selectfont 5763.668 ***} \hspace{6pt}\huxbpad{6pt}} &
\multicolumn{1}{r!{\huxvb{0, 0, 0}{0}}}{\huxtpad{6pt + 1em}\raggedleft \hspace{6pt} {\fontsize{12pt}{14.4pt}\selectfont 4045.333 ***} \hspace{6pt}\huxbpad{6pt}} \tabularnewline[-0.5pt]


\hhline{}
\arrayrulecolor{black}

\multicolumn{1}{!{\huxvb{0, 0, 0}{0}}l!{\huxvb{0, 0, 0}{0}}}{\huxtpad{6pt + 1em}\raggedright \hspace{6pt} {\fontsize{12pt}{14.4pt}\selectfont } \hspace{6pt}\huxbpad{6pt}} &
\multicolumn{1}{r!{\huxvb{0, 0, 0}{0}}}{\huxtpad{6pt + 1em}\raggedleft \hspace{6pt} {\fontsize{12pt}{14.4pt}\selectfont (13.055)\hphantom{0}\hphantom{0}\hphantom{0}} \hspace{6pt}\huxbpad{6pt}} &
\multicolumn{1}{r!{\huxvb{0, 0, 0}{0}}}{\huxtpad{6pt + 1em}\raggedleft \hspace{6pt} {\fontsize{12pt}{14.4pt}\selectfont (740.556)\hphantom{0}\hphantom{0}\hphantom{0}} \hspace{6pt}\huxbpad{6pt}} &
\multicolumn{1}{r!{\huxvb{0, 0, 0}{0}}}{\huxtpad{6pt + 1em}\raggedleft \hspace{6pt} {\fontsize{12pt}{14.4pt}\selectfont (286.205)\hphantom{0}\hphantom{0}\hphantom{0}} \hspace{6pt}\huxbpad{6pt}} \tabularnewline[-0.5pt]


\hhline{}
\arrayrulecolor{black}

\multicolumn{1}{!{\huxvb{0, 0, 0}{0}}l!{\huxvb{0, 0, 0}{0}}}{\huxtpad{6pt + 1em}\raggedright \hspace{6pt} {\fontsize{12pt}{14.4pt}\selectfont carat} \hspace{6pt}\huxbpad{6pt}} &
\multicolumn{1}{r!{\huxvb{0, 0, 0}{0}}}{\huxtpad{6pt + 1em}\raggedleft \hspace{6pt} {\fontsize{12pt}{14.4pt}\selectfont 7756.426 ***} \hspace{6pt}\huxbpad{6pt}} &
\multicolumn{1}{r!{\huxvb{0, 0, 0}{0}}}{\huxtpad{6pt + 1em}\raggedleft \hspace{6pt} {\fontsize{12pt}{14.4pt}\selectfont \hphantom{0}\hphantom{0}\hphantom{0}\hphantom{0}\hphantom{0}\hphantom{0}\hphantom{0}\hphantom{0}} \hspace{6pt}\huxbpad{6pt}} &
\multicolumn{1}{r!{\huxvb{0, 0, 0}{0}}}{\huxtpad{6pt + 1em}\raggedleft \hspace{6pt} {\fontsize{12pt}{14.4pt}\selectfont 7765.141 ***} \hspace{6pt}\huxbpad{6pt}} \tabularnewline[-0.5pt]


\hhline{}
\arrayrulecolor{black}

\multicolumn{1}{!{\huxvb{0, 0, 0}{0}}l!{\huxvb{0, 0, 0}{0}}}{\huxtpad{6pt + 1em}\raggedright \hspace{6pt} {\fontsize{12pt}{14.4pt}\selectfont } \hspace{6pt}\huxbpad{6pt}} &
\multicolumn{1}{r!{\huxvb{0, 0, 0}{0}}}{\huxtpad{6pt + 1em}\raggedleft \hspace{6pt} {\fontsize{12pt}{14.4pt}\selectfont (14.067)\hphantom{0}\hphantom{0}\hphantom{0}} \hspace{6pt}\huxbpad{6pt}} &
\multicolumn{1}{r!{\huxvb{0, 0, 0}{0}}}{\huxtpad{6pt + 1em}\raggedleft \hspace{6pt} {\fontsize{12pt}{14.4pt}\selectfont \hphantom{0}\hphantom{0}\hphantom{0}\hphantom{0}\hphantom{0}\hphantom{0}\hphantom{0}\hphantom{0}} \hspace{6pt}\huxbpad{6pt}} &
\multicolumn{1}{r!{\huxvb{0, 0, 0}{0}}}{\huxtpad{6pt + 1em}\raggedleft \hspace{6pt} {\fontsize{12pt}{14.4pt}\selectfont (14.009)\hphantom{0}\hphantom{0}\hphantom{0}} \hspace{6pt}\huxbpad{6pt}} \tabularnewline[-0.5pt]


\hhline{}
\arrayrulecolor{black}

\multicolumn{1}{!{\huxvb{0, 0, 0}{0}}l!{\huxvb{0, 0, 0}{0}}}{\huxtpad{6pt + 1em}\raggedright \hspace{6pt} {\fontsize{12pt}{14.4pt}\selectfont depth} \hspace{6pt}\huxbpad{6pt}} &
\multicolumn{1}{r!{\huxvb{0, 0, 0}{0}}}{\huxtpad{6pt + 1em}\raggedleft \hspace{6pt} {\fontsize{12pt}{14.4pt}\selectfont \hphantom{0}\hphantom{0}\hphantom{0}\hphantom{0}\hphantom{0}\hphantom{0}\hphantom{0}\hphantom{0}} \hspace{6pt}\huxbpad{6pt}} &
\multicolumn{1}{r!{\huxvb{0, 0, 0}{0}}}{\huxtpad{6pt + 1em}\raggedleft \hspace{6pt} {\fontsize{12pt}{14.4pt}\selectfont -29.650 *\hphantom{0}\hphantom{0}} \hspace{6pt}\huxbpad{6pt}} &
\multicolumn{1}{r!{\huxvb{0, 0, 0}{0}}}{\huxtpad{6pt + 1em}\raggedleft \hspace{6pt} {\fontsize{12pt}{14.4pt}\selectfont -102.165 ***} \hspace{6pt}\huxbpad{6pt}} \tabularnewline[-0.5pt]


\hhline{}
\arrayrulecolor{black}

\multicolumn{1}{!{\huxvb{0, 0, 0}{0}}l!{\huxvb{0, 0, 0}{0}}}{\huxtpad{6pt + 1em}\raggedright \hspace{6pt} {\fontsize{12pt}{14.4pt}\selectfont } \hspace{6pt}\huxbpad{6pt}} &
\multicolumn{1}{r!{\huxvb{0, 0, 0}{0}}}{\huxtpad{6pt + 1em}\raggedleft \hspace{6pt} {\fontsize{12pt}{14.4pt}\selectfont \hphantom{0}\hphantom{0}\hphantom{0}\hphantom{0}\hphantom{0}\hphantom{0}\hphantom{0}\hphantom{0}} \hspace{6pt}\huxbpad{6pt}} &
\multicolumn{1}{r!{\huxvb{0, 0, 0}{0}}}{\huxtpad{6pt + 1em}\raggedleft \hspace{6pt} {\fontsize{12pt}{14.4pt}\selectfont (11.990)\hphantom{0}\hphantom{0}\hphantom{0}} \hspace{6pt}\huxbpad{6pt}} &
\multicolumn{1}{r!{\huxvb{0, 0, 0}{0}}}{\huxtpad{6pt + 1em}\raggedleft \hspace{6pt} {\fontsize{12pt}{14.4pt}\selectfont (4.635)\hphantom{0}\hphantom{0}\hphantom{0}} \hspace{6pt}\huxbpad{6pt}} \tabularnewline[-0.5pt]


\hhline{>{\huxb{255, 255, 255}{0.4}}->{\huxb{0, 0, 0}{0.4}}->{\huxb{0, 0, 0}{0.4}}->{\huxb{0, 0, 0}{0.4}}-}
\arrayrulecolor{black}

\multicolumn{1}{!{\huxvb{0, 0, 0}{0}}l!{\huxvb{0, 0, 0}{0}}}{\huxtpad{6pt + 1em}\raggedright \hspace{6pt} {\fontsize{12pt}{14.4pt}\selectfont N} \hspace{6pt}\huxbpad{6pt}} &
\multicolumn{1}{r!{\huxvb{0, 0, 0}{0}}}{\huxtpad{6pt + 1em}\raggedleft \hspace{6pt} {\fontsize{12pt}{14.4pt}\selectfont 53940\hphantom{0}\hphantom{0}\hphantom{0}\hphantom{0}\hphantom{0}\hphantom{0}\hphantom{0}\hphantom{0}} \hspace{6pt}\huxbpad{6pt}} &
\multicolumn{1}{r!{\huxvb{0, 0, 0}{0}}}{\huxtpad{6pt + 1em}\raggedleft \hspace{6pt} {\fontsize{12pt}{14.4pt}\selectfont 53940\hphantom{0}\hphantom{0}\hphantom{0}\hphantom{0}\hphantom{0}\hphantom{0}\hphantom{0}\hphantom{0}} \hspace{6pt}\huxbpad{6pt}} &
\multicolumn{1}{r!{\huxvb{0, 0, 0}{0}}}{\huxtpad{6pt + 1em}\raggedleft \hspace{6pt} {\fontsize{12pt}{14.4pt}\selectfont 53940\hphantom{0}\hphantom{0}\hphantom{0}\hphantom{0}\hphantom{0}\hphantom{0}\hphantom{0}\hphantom{0}} \hspace{6pt}\huxbpad{6pt}} \tabularnewline[-0.5pt]


\hhline{}
\arrayrulecolor{black}

\multicolumn{1}{!{\huxvb{0, 0, 0}{0}}l!{\huxvb{0, 0, 0}{0}}}{\huxtpad{6pt + 1em}\raggedright \hspace{6pt} {\fontsize{12pt}{14.4pt}\selectfont R2} \hspace{6pt}\huxbpad{6pt}} &
\multicolumn{1}{r!{\huxvb{0, 0, 0}{0}}}{\huxtpad{6pt + 1em}\raggedleft \hspace{6pt} {\fontsize{12pt}{14.4pt}\selectfont 0.849\hphantom{0}\hphantom{0}\hphantom{0}\hphantom{0}} \hspace{6pt}\huxbpad{6pt}} &
\multicolumn{1}{r!{\huxvb{0, 0, 0}{0}}}{\huxtpad{6pt + 1em}\raggedleft \hspace{6pt} {\fontsize{12pt}{14.4pt}\selectfont 0.000\hphantom{0}\hphantom{0}\hphantom{0}\hphantom{0}} \hspace{6pt}\huxbpad{6pt}} &
\multicolumn{1}{r!{\huxvb{0, 0, 0}{0}}}{\huxtpad{6pt + 1em}\raggedleft \hspace{6pt} {\fontsize{12pt}{14.4pt}\selectfont 0.851\hphantom{0}\hphantom{0}\hphantom{0}\hphantom{0}} \hspace{6pt}\huxbpad{6pt}} \tabularnewline[-0.5pt]


\hhline{>{\huxb{0, 0, 0}{0.8}}->{\huxb{0, 0, 0}{0.8}}->{\huxb{0, 0, 0}{0.8}}->{\huxb{0, 0, 0}{0.8}}-}
\arrayrulecolor{black}

\multicolumn{4}{!{\huxvb{0, 0, 0}{0}}l!{\huxvb{0, 0, 0}{0}}}{\huxtpad{6pt + 1em}\raggedright \hspace{6pt} {\fontsize{12pt}{14.4pt}\selectfont  *** p $<$ 0.001;  ** p $<$ 0.01;  * p $<$ 0.05.} \hspace{6pt}\huxbpad{6pt}} \tabularnewline[-0.5pt]


\hhline{}
\arrayrulecolor{black}
\end{tabular}
\end{threeparttable}\par\end{centerbox}

\end{table}
 

FYI - R also recently introduced the gt package, which is worthwhile
exploring too.

\hypertarget{lists}{%
\section{Lists}\label{lists}}

To add lists, simply using the following notation

\begin{itemize}
\item
  This is really simple

  \begin{itemize}
  \tightlist
  \item
    Just note the spaces here - writing in R you have to sometimes be
    pedantic about spaces\ldots{}
  \end{itemize}
\item
  Note that Rmarkdown notation removes the pain of defining
  \LaTeX environments!
\end{itemize}

\hypertarget{conclusion}{%
\section{\texorpdfstring{Conclusion
\label{con}}{Conclusion }}\label{conclusion}}

I hope you find this template useful. Remember, stackoverflow is your
friend - use it to find answers to questions. Feel free to write me a
mail if you have any questions regarding the use of this package. To
cite this package, simply type citation(``Texevier'') in Rstudio to get
the citation for \protect\hyperlink{ref-Texevier}{Katzke}
(\protect\hyperlink{ref-Texevier}{2017}) (Note that uncited references
in your bibtex file will not be included in References).

\newpage

\hypertarget{references}{%
\section*{References}\label{references}}
\addcontentsline{toc}{section}{References}

\hypertarget{refs}{}
\begin{CSLReferences}{1}{0}
\leavevmode\hypertarget{ref-fama1997}{}%
Fama, E.F. \& French, K.R. 1997. Industry costs of equity. \emph{Journal
of financial economics}. 43(2):153--193.

\leavevmode\hypertarget{ref-grinold2000}{}%
Grinold, R.C. \& Kahn, R.N. 2000. Active portfolio management.

\leavevmode\hypertarget{ref-Texevier}{}%
Katzke, N.F. 2017. \emph{{Texevier}: {P}ackage to create elsevier
templates for rmarkdown}. Stellenbosch, South Africa: Bureau for
Economic Research.

\leavevmode\hypertarget{ref-stat}{}%
RSA. 2021. \emph{Latest vaccine statistics}. Department of Health.
{[}Online{]}, Available:
\url{https://sacoronavirus.co.za/latest-vaccine-statistics/}.

\end{CSLReferences}

\hypertarget{appendix}{%
\section*{Appendix}\label{appendix}}
\addcontentsline{toc}{section}{Appendix}

\hypertarget{appendix-a}{%
\subsection*{Appendix A}\label{appendix-a}}
\addcontentsline{toc}{subsection}{Appendix A}

Some appendix information here

\hypertarget{appendix-b}{%
\subsection*{Appendix B}\label{appendix-b}}
\addcontentsline{toc}{subsection}{Appendix B}

\bibliography{Tex/ref}





\end{document}
