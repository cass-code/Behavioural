\documentclass[11pt,preprint, authoryear]{elsarticle}

\makeatletter
\renewcommand\@biblabel[1]{}
\makeatother

\usepackage{lmodern}
%%%% My spacing
\usepackage{setspace}
\setstretch{1.2}
\DeclareMathSizes{12}{14}{10}{10}

% Wrap around which gives all figures included the [H] command, or places it "here". This can be tedious to code in Rmarkdown.
\usepackage{float}
\let\origfigure\figure
\let\endorigfigure\endfigure
\renewenvironment{figure}[1][2] {
    \expandafter\origfigure\expandafter[H]
} {
    \endorigfigure
}

\let\origtable\table
\let\endorigtable\endtable
\renewenvironment{table}[1][2] {
    \expandafter\origtable\expandafter[H]
} {
    \endorigtable
}


\usepackage{ifxetex,ifluatex}
\usepackage{fixltx2e} % provides \textsubscript
\ifnum 0\ifxetex 1\fi\ifluatex 1\fi=0 % if pdftex
  \usepackage[T1]{fontenc}
  \usepackage[utf8]{inputenc}
\else % if luatex or xelatex
  \ifxetex
    \usepackage{mathspec}
    \usepackage{xltxtra,xunicode}
  \else
    \usepackage{fontspec}
  \fi
  \defaultfontfeatures{Mapping=tex-text,Scale=MatchLowercase}
  \newcommand{\euro}{€}
\fi

\usepackage{amssymb, amsmath, amsthm, amsfonts}

\def\bibsection{\section*{References}} %%% Make "References" appear before bibliography


\usepackage[round]{natbib}

\usepackage{longtable}
\usepackage[margin=2.3cm,bottom=2cm,top=2.5cm, includefoot]{geometry}
\usepackage{fancyhdr}
\usepackage[bottom, hang, flushmargin]{footmisc}
\usepackage{graphicx}
\numberwithin{equation}{section}
\numberwithin{figure}{section}
\numberwithin{table}{section}
\setlength{\parindent}{0cm}
\setlength{\parskip}{1.3ex plus 0.5ex minus 0.3ex}
\usepackage{textcomp}
\renewcommand{\headrulewidth}{0.2pt}
\renewcommand{\footrulewidth}{0.3pt}

\usepackage{array}
\newcolumntype{x}[1]{>{\centering\arraybackslash\hspace{0pt}}p{#1}}

%%%%  Remove the "preprint submitted to" part. Don't worry about this either, it just looks better without it:
\makeatletter
\def\ps@pprintTitle{%
  \let\@oddhead\@empty
  \let\@evenhead\@empty
  \let\@oddfoot\@empty
  \let\@evenfoot\@oddfoot
}
\makeatother

 \def\tightlist{} % This allows for subbullets!

\usepackage{hyperref}
\hypersetup{breaklinks=true,
            bookmarks=true,
            colorlinks=true,
            citecolor=blue,
            urlcolor=blue,
            linkcolor=blue,
            pdfborder={0 0 0}}


% The following packages allow huxtable to work:
\usepackage{siunitx}
\usepackage{multirow}
\usepackage{hhline}
\usepackage{calc}
\usepackage{tabularx}
\usepackage{booktabs}
\usepackage{caption}


\newenvironment{columns}[1][]{}{}

\newenvironment{column}[1]{\begin{minipage}{#1}\ignorespaces}{%
\end{minipage}
\ifhmode\unskip\fi
\aftergroup\useignorespacesandallpars}

\def\useignorespacesandallpars#1\ignorespaces\fi{%
#1\fi\ignorespacesandallpars}

\makeatletter
\def\ignorespacesandallpars{%
  \@ifnextchar\par
    {\expandafter\ignorespacesandallpars\@gobble}%
    {}%
}
\makeatother

\newlength{\cslhangindent}
\setlength{\cslhangindent}{1.5em}
\newenvironment{CSLReferences}%
  {\setlength{\parindent}{0pt}%
  \everypar{\setlength{\hangindent}{\cslhangindent}}\ignorespaces}%
  {\par}


\urlstyle{same}  % don't use monospace font for urls
\setlength{\parindent}{0pt}
\setlength{\parskip}{6pt plus 2pt minus 1pt}
\setlength{\emergencystretch}{3em}  % prevent overfull lines
\setcounter{secnumdepth}{5}

%%% Use protect on footnotes to avoid problems with footnotes in titles
\let\rmarkdownfootnote\footnote%
\def\footnote{\protect\rmarkdownfootnote}
\IfFileExists{upquote.sty}{\usepackage{upquote}}{}

%%% Include extra packages specified by user

%%% Hard setting column skips for reports - this ensures greater consistency and control over the length settings in the document.
%% page layout
%% paragraphs
\setlength{\baselineskip}{12pt plus 0pt minus 0pt}
\setlength{\parskip}{12pt plus 0pt minus 0pt}
\setlength{\parindent}{0pt plus 0pt minus 0pt}
%% floats
\setlength{\floatsep}{12pt plus 0 pt minus 0pt}
\setlength{\textfloatsep}{20pt plus 0pt minus 0pt}
\setlength{\intextsep}{14pt plus 0pt minus 0pt}
\setlength{\dbltextfloatsep}{20pt plus 0pt minus 0pt}
\setlength{\dblfloatsep}{14pt plus 0pt minus 0pt}
%% maths
\setlength{\abovedisplayskip}{12pt plus 0pt minus 0pt}
\setlength{\belowdisplayskip}{12pt plus 0pt minus 0pt}
%% lists
\setlength{\topsep}{10pt plus 0pt minus 0pt}
\setlength{\partopsep}{3pt plus 0pt minus 0pt}
\setlength{\itemsep}{5pt plus 0pt minus 0pt}
\setlength{\labelsep}{8mm plus 0mm minus 0mm}
\setlength{\parsep}{\the\parskip}
\setlength{\listparindent}{\the\parindent}
%% verbatim
\setlength{\fboxsep}{5pt plus 0pt minus 0pt}



\begin{document}



%titlepage
\thispagestyle{empty}
\begin{center}
\begin{minipage}{0.75\linewidth}
    \centering
%Entry1
    {\uppercase{\huge COVID-19 Vaccine Lottery Field Experiment\par}}
    \vspace{2cm}
%Author's name
    {\LARGE \textbf{Cassandra Pengelly | 20346212}\par}
    \vspace{1cm}
%University logo
\begin{center}
    \includegraphics[width=0.9\linewidth]{Tex/logo.png}
\end{center}
\vspace{1cm}
%Supervisor's Details
\begin{center}
    {\LARGE Professor R. Burger\par}
    \vspace{1cm}
%Degree
    {\LARGE Behavioural Economics 871 Essay\par}
    \vspace{1cm}
%Institution
    {\LARGE 15 October 2021\par}
    \vspace{1cm}
%Date
    {\large }
%More
    {\normalsize }
%More
    {\normalsize }
\end{center}
\end{minipage}
\end{center}
\clearpage


\begin{frontmatter}  %

\title{}

% Set to FALSE if wanting to remove title (for submission)


\vspace{1cm}





\vspace{0.5cm}

\end{frontmatter}


\renewcommand{\contentsname}{Table of Contents}
{\tableofcontents}

%________________________
% Header and Footers
%%%%%%%%%%%%%%%%%%%%%%%%%%%%%%%%%
\pagestyle{fancy}
\chead{}
\rhead{}
\lfoot{}
\rfoot{\footnotesize Page \thepage}
\lhead{}
%\rfoot{\footnotesize Page \thepage } % "e.g. Page 2"
\cfoot{}

%\setlength\headheight{30pt}
%%%%%%%%%%%%%%%%%%%%%%%%%%%%%%%%%
%________________________

\headsep 35pt % So that header does not go over title




\newpage

\hypertarget{introduction-492}{%
\section{\texorpdfstring{Introduction \label{Introduction}
492}{Introduction  492}}\label{introduction-492}}

Coronavirus disease 2019 (COVID-19) has caused the largest public health
crisis, and economic disaster of the 21st century so far
(\protect\hyperlink{ref-bad}{Kadkhoda, 2021: 471}). According to the
\protect\hyperlink{ref-who}{World Health Organisation}
(\protect\hyperlink{ref-who}{2021}), COVID-19 has directly resulted in
4,859,277 deaths worldwide\footnote{As of 13 October 2021}, and its
impact on the global economy has been severe
(\protect\hyperlink{ref-bank}{World Bank, 2020}). Policymakers are in
urgent need of evidence-based strategies to contain the pandemic. As
\protect\hyperlink{ref-immun}{Fontanet \& Cauchemez}
(\protect\hyperlink{ref-immun}{2020: 583}) report, one critical
mechanism through which epidemics are controlled is through herd
immunity. Herd immunity arises when a sufficiently large proportion of
the population achieves individual immunity to an infectious disease
such that the transmission chain of the disease is halted
(\protect\hyperlink{ref-bad}{Kadkhoda, 2021: 471}). One method of
establishing herd immunity is through vaccination programs
(\protect\hyperlink{ref-immun}{Fontanet \& Cauchemez, 2020: 583}).
Preliminary empirical evidence shows that infection detection and
vaccination strategies could be critical tools in subduing COVID-19, for
example the study by \protect\hyperlink{ref-erad}{Aldila, Samiadji,
Simorangkir, Khosnaw \& Shahzad} (\protect\hyperlink{ref-erad}{2021})
finds COVID-19 vaccines to be effective in Jakarta, Indonesia. Leaders
of other countries have also implemented vaccine programs and the
\protect\hyperlink{ref-who}{World Health Organisation}
(\protect\hyperlink{ref-who}{2021}) reports that a total of
6,364,021,792 vaccine doses have been administered globally as of 10
October 2021.

In line with the international approach, South Africa has opted to issue
vaccines in addition to social distancing and lockdown measures. The
South African government has stated that it aims to have 67\% of the
population vaccinated by the end of 2021
(\protect\hyperlink{ref-herd}{Department of Health, 2021a}). However,
many South Africans are hesitant to get the COVID-19 vaccine; the
\protect\hyperlink{ref-stat}{Department of Health}
(\protect\hyperlink{ref-stat}{2021b}) records that only 25\% of South
Africa's adult population has been fully vaccinated as of 11 October
2021. \protect\hyperlink{ref-cram}{Burger, Maughan-Brown, Köhler,
English \& Tameris} (\protect\hyperlink{ref-cram}{2021}) compiled a
report, based on data from wave 5 of the NIDS-CRAM survey, and find that
that around 20\% of South Africans are concerned that COVID-19 vaccines
are not safe. The report also shows that relatively few people in the
sample registered to be vaccinated within two months after registration
opened. \protect\hyperlink{ref-cram}{Burger, Maughan-Brown, Köhler,
English \& Tameris} (\protect\hyperlink{ref-cram}{2021}) conclude that a
significant portion of South Africans still have to be convinced to take
the vaccine\footnote{A sentiment report from
  \protect\hyperlink{ref-report}{Department of Health}
  (\protect\hyperlink{ref-report}{2021c}) shows an interesting break
  down of different communities' beliefs surrounding the vaccine}. While
there are different ways to encourage vaccine uptake, such as mandatory
vaccination or lump-sum transfers, insights from behaviourial economics
could provide a more cost-effective solution: a vaccine lottery.

This essay\footnote{This essay was written in R using the package
  Texevier by \protect\hyperlink{ref-Texevier}{Katzke}
  (\protect\hyperlink{ref-Texevier}{2017})} proposes a field experiment
to investigate whether a vaccine lottery could improve vaccination rates
in South Africa. The experiment explores the effect of three different
lottery types - standard, regret and referral - on the take up of
vaccines. This essay is structured as follows. Section \ref{lit} briefly
reviews the relevant literature on behavioural economics and health
incentives. Section \ref{context} elaborates on the South African
context. Section \ref{design} describes the design of the experiment and
outlines the three types of treatment groups. Section \ref{treatment}
discusses how the treatment will be administered and how the data will
be collected. Lastly, section \ref{pre} gives a pre-analysis plan of the
empirical analysis that will be performed on the data, and the final
section (\ref{con}) concludes.

• a clear statement of the research question and motivation for why this
is interesting and important;

\hypertarget{behavioural-economics-and-health-incentives}{%
\section{\texorpdfstring{Behavioural Economics and Health Incentives
\label{lit}}{Behavioural Economics and Health Incentives }}\label{behavioural-economics-and-health-incentives}}

Health professionals and policymakers are increasingly turning to
behavioural economics to understand how people make health decisions and
how behavioural insights can be used to improve public health outcomes
(\protect\hyperlink{ref-health}{Loewenstein, Asch, Friedman, Melichar \&
Volpp, 2012: 1}). While neoclassical economics assumes that people are
perfectly rational agents, behavioural economics relaxes this assumption
and uses psychology and economic theory to create more realistic models
of human decision-making (\protect\hyperlink{ref-rabin}{Rabin, 2002}).
As Khaneman and Tversky show, people are subject to certain biases and
often make use of heuristics in their decision-making process, which can
lead to predictable errors in judgment
\protect\hyperlink{ref-prospect}{Kahneman \& Tversky}
(\protect\hyperlink{ref-prospect}{1979}). A large literature has
developed in the field of behavioural economics to investigate how these
biases can be combated and sub-optimal choices overcome to improve
welfare. \protect\hyperlink{ref-nudge}{Thaler \& Sunstein}
(\protect\hyperlink{ref-nudge}{2008}) introduced the idea of a
nudge\footnote{Nudge: an intervention that alters behaviour towards a
  desired action. In order for an intervention to qualify as a nudge, it
  should be cheap and easy to avoid
  (\protect\hyperlink{ref-nudge}{Thaler \& Sunstein, 2008}).} as a way
to guide people to make better choices. For example,
\protect\hyperlink{ref-nudge}{Thaler \& Sunstein}
(\protect\hyperlink{ref-nudge}{2008: 176}) proposed changing the default
option for organ donation in America to opt-in as opposed to explicit
consent. They found that many people who were willing to be organ donors
did not take the necessary steps, and that the registration process
appeared to deter otherwise willing donors. By changing the choice
architecture to presumed consent (with an easy opt-out option), there
would be more registered donors and more lives saved.

A similar type of nudge can be applied to flu vaccines. A study
conducted by \protect\hyperlink{ref-opt}{Chapman, Li, Colby \& Yoon}
(\protect\hyperlink{ref-opt}{2010}) on a sample of 478 participants
(split into two treatment groups of 239 each) found that vaccination
rates increased by 36\% under an opt-in default than under an opt-out
condition. \protect\hyperlink{ref-flu}{Madrian}
(\protect\hyperlink{ref-flu}{2014: 9}) proposes other interventions for
promoting flu vaccination, such as moving the vaccination clinic to a
central location for visibility purposes, reminding people more
frequently to get vaccinated, and encouraging people to plan the time
and location they will receive their flu vaccination. The last two
nudges have been, to some extent, implemented in South Africa for the
COVID-19 vaccine: the government has sent out SMS's encouraging South
Africans to get vaccinated, and a self registration portal has been set
up for citizens to enroll in the Electronic Vaccination Data System
(EVDS) (\protect\hyperlink{ref-evds}{Republic of South Africa, 2021}).
This portal can be seen as a type of commitment device, in addition to
being a data collection mechanism.

Behavioural economic theory and empirical studies suggest that lotteries
can be a useful device for public health interventions. A lottery system
can be a cost-effective mechanism for changing behaviour compared to
direct transfers because people tend to overweight small probabilities,
which is a key insight from the work by
\protect\hyperlink{ref-prospect}{Kahneman \& Tversky}
(\protect\hyperlink{ref-prospect}{1979: 286}) on prospect theory. Due to
this nonlinear probability weighting, an individual overestimates her
chances of winning a lottery. \protect\hyperlink{ref-hiv}{Björkman
Nyqvist, Corno, Walque \& Svensson} (\protect\hyperlink{ref-hiv}{2018})
ran an experiment in Lesotho, where participants were entered into a
financial lottery and they could win a cash prize if they tested
negative for sexually transmitted infections. In spite of the fact that
the lottery had low expected payments, HIV incidence decreased by 21.4\%
over two years as a result of the intervention.
\protect\hyperlink{ref-hiv}{Björkman Nyqvist, Corno, Walque \& Svensson}
(\protect\hyperlink{ref-hiv}{2018}) found that the lottery incentive
worked particularly well at targeting participants who were more prone
to risky sexual behaviour. This supports the theory that risk-seeking
individuals value lotteries more. In a related study,
\protect\hyperlink{ref-hiv}{Björkman Nyqvist, Corno, Walque \& Svensson}
(\protect\hyperlink{ref-hiv}{2018}) investigated the impact of a lottery
intervention on the health choices of men living with HIV in South
Africa. The paper found that the participants who were eligible for the
lottery started antiretroviral therapy sooner than those in the control
group (\protect\hyperlink{ref-hiv}{Björkman Nyqvist, Corno, Walque \&
Svensson, 2018: 49}). This provides some indication that a vaccine
lottery in South Africa could potentially be effective.

Drawing further on prospect theory, the concepts of loss aversion,
reference dependence and regret avoidance can also be included in health
interventions through a ``regret lottery''.
\protect\hyperlink{ref-prospect}{Kahneman \& Tversky}
(\protect\hyperlink{ref-prospect}{1979}) describe loss aversion as a
cognitive bias whereby people experience losses as more painful than the
pleasure they receive from an equivalent gain. Thus, people are more
willing to take on risk in order to avoid a loss, and are less risk
seeking when pursuing gain (\protect\hyperlink{ref-prospect}{Kahneman \&
Tversky, 1979: 268}). Reference dependence follows on from loss aversion
and suggests that people define gains and losses relative to a reference
point (\protect\hyperlink{ref-ref}{Tversky \& Kahneman, 1991: 1039}).
People are also subject to regret avoidance; where there is a
significant emotional cost attached to regret and thus people will make
decisions to avoid regretting alternative decisions in the future
(\protect\hyperlink{ref-regret}{Bailey \& Kinerson, 2005}).

A regret lottery takes advantage of these three principles by entering
all participants into a lottery but the winner can only claim the prize
contingent on some condition. If this condition is not met, a new winner
is selected. By entering all participants, people's reference point is
shifted to ``I have a chance at winning the lottery''. However, if a
person is not eligible to claim the prize because he does not meet the
required condition, he feels he is losing the chance to claim the prize.
He is more likely to try and meet the condition in order to minimise the
pain of this loss. Additionally, he will want to avoid the regret that
would come from having missed the opportunity to claim the prize.

Several empirical papers investigate how ``regret lotteries'' can
improve health behaviours. \protect\hyperlink{ref-adhere}{Humphrey,
Small, Jensen, Volpp, Asch, Zhu \& Troxel}
(\protect\hyperlink{ref-adhere}{2019}) analysed the effect of a daily
regret lottery on cholesterol-lowering, and heart medication adherence.
Every day, members of the treatment group chose a two-digit number. If a
participant's number matched at least one digit of the randomly drawn
lottery number, she was eligible to win a cash prize. However, she could
only claim her winnings if she had taken her relevant prescribed
medication the previous day. The results of the paper show that the
treatment group better adhered to their medication regime than the
control group. The authors also noted that participants appeared to be
engaged even after 6 months of the program. In a different study,
\protect\hyperlink{ref-regr}{Husain, Diaz, Schwartz, Parsons, Burg,
Davidson \& Kronish} (\protect\hyperlink{ref-regr}{2019}) found that
implementing a weekly electronic regret lottery increased adherence to a
self-monitoring study protocol.

Some states in America have attempted to run regret lotteries to
encourage people to get vaccinated against COVID-19. An unpublished
paper by \protect\hyperlink{ref-duck}{Gandhi, Milkman, Ellis, Graci,
Gromet, Mobarak, Buttenheim, Duckworth, Pope, Stanford, Thaler \& Volpp}
(\protect\hyperlink{ref-duck}{2021}) evaluates the effect of weekly
regret lotteries in Philadelphia. After giving away around \$400,000 in
cash prizes to residents, the authors did not find convincing evidence
that the regret lotteries significantly increased first-dose vaccination
rates for the treatment groups. Although, the authors do acknowledge
that there were some design flaws in the experiment that could be
clouding the results (\protect\hyperlink{ref-duck}{Gandhi, Milkman,
Ellis, Graci, Gromet, Mobarak, Buttenheim, Duckworth, Pope, Stanford,
Thaler \& Volpp, 2021: 3}). Similarly disappointing results were found
in Ohio, where randomly selected vaccine recipients could receive up to
\$1 million. The study by \protect\hyperlink{ref-ohio}{Walkey, Law \&
Bosch} (\protect\hyperlink{ref-ohio}{2021}) found no increase in
COVID-19 vaccination rates following the lottery incentive. To the best
of my knowledge, there have been no studies on COVID-19 vaccine
lotteries in South Africa\footnote{It should be noted, however, that
  First National Bank is running a COVID-19 vaccine lottery for FNB
  customers, with total cash prizes amounting to R18 million} - a gap
which this experiment is intended to fill. This experiment would also
contribute to the empirical literature on lotteries as public health
interventions and the integration of behavioural economics into public
policy.

\hypertarget{the-south-african-context}{%
\section{\texorpdfstring{The South African Context
\label{context}}{The South African Context }}\label{the-south-african-context}}

\protect\hyperlink{ref-lotto}{National Lotteries Commission}
(\protect\hyperlink{ref-lotto}{2019}) investigated the lottery habits
and attitudes of South Africans based on a sample of 3,090 households
randomly distributed across the country. The integrated report shows
that around 35\% participated in lottery activities in the year
preceding the study (\protect\hyperlink{ref-lotto}{National Lotteries
Commission, 2019: 78--79}). This suggests that a significant amount of
South Africans engage in lotteries, which supports the implementation of
a vaccine lottery, as there is evidence of ``demand'' for lotteries. The
average amount spent on lottery activities was R156 per month, whereas
the average monthly winnings of lottery schemes was R110. A rational
economic agent would recognize that the expected value for playing
lotteries is negative (\(-\)R46) and not buy lottery tickets. The fact
that people still play the lotto despite a negative expected value
suggests that South Africans may not be rational, and behavioural
economic insights are applicable. The gross revenue from gambling
activities, excluding the National Lottery, amounted to R32.7 billion
for the 2019 financial year (\protect\hyperlink{ref-gamble}{National
Gambling Board, 2020: 3}). This represents 0.64\% of South Africa's 2019
nominal GDP (R5.1 trillion) (\protect\hyperlink{ref-statsa}{Statistics
South Africa, 2020: 8}), which suggests that gambling is a lucrative
market in South Africa and a lotto device could be an appropriate tool
for incentivising behaviour.

There vaccines which have been approved by the
\protect\hyperlink{ref-sah}{South African Health Products Regulatory
Authority} (\protect\hyperlink{ref-sah}{2021}) for use in South Africa
include: Johnson \& Johnson (J\&J), Pfizer and AstraZeneca. COVID-19
vaccines are available free of charge, but citizens have to register on
the EVID portal before receiving their dose. Due to supply constraints,
vaccines were rolled out in phases and are currently available to
individuals over the age of 18 only. The J\&J vaccine only requires 1
dose; the Pfizer and AstraZeneca vaccines require two doses each. For
this experiment, \emph{fully vaccinated} refers to an individual who has
had the maximum required doses of any vaccine. \emph{Vaccinated} refers
to an individual who has had 1 shot of any vaccine.

According to \protect\hyperlink{ref-stat}{Department of Health}
(\protect\hyperlink{ref-stat}{2021b}), 34\% of South Africans are
vaccinated, while only 25\% are fully vaccinated. These low vaccination
numbers Table \ref{tab1} below gives the breakdown of vaccination rates
by province.

\begin{table}[H]
\centering
\begin{tabular}{llll}
  \toprule
Province & Total Adults Vaccinated & Adult Population & Percentage Vaccinated \\ 
  \midrule
Eastern Cape & 1 603 045 & 4 099 543 & 39\% \\ 
  Free State & 735 696 & 1 914 521 & 38\% \\ 
  Gauteng & 3 523 373 & 11 311 326 & 31\% \\ 
  KwaZulu-Natal & 2 170 526 & 7 219 795 & 30\% \\ 
  Limpopo & 1 437 846 & 3 695 801 & 39\% \\ 
  Mpumalanga & 831 759 & 3 039 520 & 27\% \\ 
  North West & 835 206 & 2 693 247 & 31\% \\ 
  Northern Cape & 290 962 & 847 545 & 34\% \\ 
  Western Cape & 2 141 933 & 4 976 903 & 43\% \\ 
  Total & 13 570 346 & 39 798 201 & 34\% \\ 
   \bottomrule
\end{tabular}
\caption{Vaccination Statistics \label{tab1}} 
\end{table}

\hypertarget{experiment-design}{%
\section{\texorpdfstring{Experiment Design
\label{design}}{Experiment Design }}\label{experiment-design}}

\hypertarget{sample}{%
\subsection{Sample}\label{sample}}

The sample used will be the same sample used for the Coronavirus Rapid
Mobile Survey (NIDS-CRAM). NIDS-CRAM was created in response to the
pandemic as a way to build a representative data set of the South
African population to inform decision-making
(\protect\hyperlink{ref-nids}{Ingle, 2021}). Thus far, there have been
five waves of NIDS-CRAM surveys, with wave 5 comprising a sample of
5,862 people being surveyed during 6 April to 11 May 2021
(\protect\hyperlink{ref-nids}{Ingle, 2021: 14}). Wave 3 consisted of a
total of 8,157 potential participants, of which 6,130 were successfully
interviewed. For the purposes of this field experiment, the 8,157 people
from Wave 3 will be contacted and asked to participate in the
experiment. It seems reasonable to expect between 5,500 and 6,200 people
to participate, given the previous rates of attrition experienced in
NIDS-CRAM.

If we assume a conservative sample size of 5,500 and a vaccination
proportion of 34\%, then there will be 3630 eligible participants for
the study, since we are interested in unvaccinated individuals. This
sample will be split into 4 groups of equal size (907), and randomized
according to technique proposed by \protect\hyperlink{ref-random}{Duflo,
Glennerster \& Kremer} (\protect\hyperlink{ref-random}{2007}). One group
is the control group, where individuals will not be entered into any
vaccine lottery. The other 3 groups are will receive different lottery
treatments, which are explained below (\ref{group}). There will be 3
monthly lotteries for each treatment group, which will run
simultaneously. Thus, in total there will be 9 lotteries for this field
experiment, with 3 lotteries every month. Each lottery winner will
receive a cash prize of R1,000,000.

\hypertarget{treatment-groups}{%
\subsection{\texorpdfstring{Treatment Groups
\label{group}}{Treatment Groups }}\label{treatment-groups}}

For the first treatment group (group 1), if an individual has received a
vaccination shot within a given month, she will be entered into that
month's vaccine lottery. At the end of the month, a winner is randomly
selected from the lottery pool. Once it is verified that the winner has
been vaccinated, she will be privately contacted and will receive a cash
prize. The second lottery is a regret lottery. Every individual in the
sample (group 2) is entered into a monthly lottery but an individual may
only claim her prize if she has been vaccinated (i.e.~had at least 1
shot of any vaccine). At month end, a winner is randomly drawn from the
lottery pool. If the winner has been vaccinated, she will be privately
notified and will receive a cash prize. If the winner has not been
vaccinated, she will receive a ``regret'' message stating that she would
have won the cash prize if she had been vaccinated.

The final treatment is a ``referral lottery''. An individual is entered
into the monthly lottery if 2 conditions are met: he is vaccinated, and
he refers a friend to get vaccinated and the friend gets vaccinated.
Both the individual from the sample and his friend are entered into the
lottery. At the end of the month, a winner is selected, and once he and
his referral partner are verified to be vaccinated, he will be privately
informed. An individual can refer more than friend to be vaccinated in
any given month. However, only individuals from the treatment group can
refer friends (i.e.~people outside of group 3 will not be entered into
the lottery for referring others to get vaccinated). Group 3
participants may not refer any person in the control group or in group 1
or group 2 (to avoid contamination).

Whenever a lottery is won, it will be announced via SMS to all
participants in the relevant treatment group. The amount of the lottery
prize and the winner's province will also be included in the SMS. This
serves as a reminder of how large the cash prize is, and including the
winner's province makes winning seem more tangible/possible. Group 2's
SMS will include a reminder that only vaccinated individuals are
eligible to win the lottery. Group 3's SMS will include a reminder that
participants can refer as many friends as they like to be eligible for
the following month's lottery. All SMS's will end with: ``Thank you for
vaccinating and keeping our country safe!'' as one final nudge to
encourage/guilt participants to vaccinate.

Table \ref{tab2} summarises the different treatments administered.

\begin{table}[H]
\centering
\begin{tabular}{ll}
  \toprule
Group & Treatment \\ 
  \midrule
Control & No lottery \\ 
  Group 1 & Individual is entered into a lottery once they are vaccinated \\ 
  Group 2 & Everyone in the group is entered into a lottery; only vaccinated individuals can \\ 
   & claim the prize \\ 
  Group 3 & Individual is entered if she is vaccinated and refers a friend, who gets vaccinated \\ 
   \bottomrule
\end{tabular}
\caption{Treatment Summary \label{tab2}} 
\end{table}

\hypertarget{theory-of-change}{%
\subsection{Theory of Change}\label{theory-of-change}}

Target Group The lucky draw is anticipated to attract people who are
risk-on (they enjoy gambling, and are less worried about getting
vaccinated), and poorer individuals for whom winning money is more
attractive. These target groups are desirable as they are less likely to
get the vaccine, and the government would like to maximise the number of
vaccinated people. Additionally, if there are individuals who want to be
vaccinated but procrastinate getting the vaccine (e.g.~naïve hyperbolic
discounters), setting a deadline for the lucky draw could increase the
utility of getting the vaccine earlier enough to overcome the
procrastination problem. There is no downside or extra cost for having
people enter the lucky draw who would otherwise still have got the
vaccine.

\hypertarget{treatment-and-data}{%
\section{\texorpdfstring{Treatment and Data
\label{treatment}}{Treatment and Data }}\label{treatment-and-data}}

The two main institutions for this field experiment would be the South
African government, the Department of Health in particular, and

• an explanation of how the treatments will be administered and data
gathered (including proposed partner institutions);

The CRAM survey exists to provide monthly nationally-representative data
on topics such as unemployment, household income, child hunger and
access to government grants.

funding

Partner with NIDS, department of health, vaccine administer, funding
data collected at vaccine add 3 questions to NIDS data cram

\hypertarget{pre-analysis-plan}{%
\section{\texorpdfstring{Pre-analysis plan
\label{pre}}{Pre-analysis plan }}\label{pre-analysis-plan}}

• a pre-analysis plan of the empirical analysis that will be performed
on the data.

\url{https://jamanetwork.com/journals/jama/fullarticle/2781792} - method
for analysis

Data Collection There is a data collection system already set up at the
vaccination sites so this extra data point would not be difficult to
collect within the current tracking system. Depending on costs, the
referral friend could be sent an SMS thanking her for caring about
others and getting them vaccinated, and letting her know that she has
been entered into the draw. This is a positive reinforcement technique
and shows people that the government is following up on their promise.
This acknowledgement and transparency is expected to encourage more
referrals. Once the lucky draw has been concluded, the data can be
analysed, the purpose of which is to uncover whether the nudge increased
vaccinations.

\hfill

\hypertarget{conclusion}{%
\section{\texorpdfstring{Conclusion
\label{con}}{Conclusion }}\label{conclusion}}

If the experiment results provide enough evidence that a vaccine lottery
is effective, the EVDS nudge is to rollout to whole country and possible

\newpage

\hypertarget{references}{%
\section*{References}\label{references}}
\addcontentsline{toc}{section}{References}

\hypertarget{refs}{}
\begin{CSLReferences}{1}{0}
\leavevmode\hypertarget{ref-erad}{}%
Aldila, D., Samiadji, B.M., Simorangkir, G.M., Khosnaw, S.H. \& Shahzad,
M. 2021. Impact of early detection and vaccination strategy in COVID-19
eradication program in jakarta, indonesia. \emph{BMC Research Notes}.
14(1):1--7.

\leavevmode\hypertarget{ref-regret}{}%
Bailey, J.J. \& Kinerson, C. 2005. Regret avoidance and risk tolerance.
\emph{Journal of Financial Counseling and Planning}. 16(1):23.

\leavevmode\hypertarget{ref-hiv}{}%
Björkman Nyqvist, M., Corno, L., Walque, D. de \& Svensson, J. 2018.
Incentivizing safer sexual behavior: Evidence from a lottery experiment
on HIV prevention. \emph{American Economic Journal: Applied Economics}.
10(3):287--314.

\leavevmode\hypertarget{ref-cram}{}%
Burger, R., Maughan-Brown, B., Köhler, T., English, R. \& Tameris, M.
2021. \emph{A shot in the arm for south africa - increased openness to
accepting a COVID-19 vaccine: Evidence from NIDS-CRAM waves 4 and 5}.
National Income Dynamics Study (NIDS) -- Coronavirus Rapid Mobile Survey
(CRAM). {[}Online{]}, Available: \url{https://cramsurvey.org/reports/}.

\leavevmode\hypertarget{ref-opt}{}%
Chapman, G.B., Li, M., Colby, H. \& Yoon, H. 2010. Opting in vs opting
out of influenza vaccination. \emph{JAMA}. 304(1):43--44.

\leavevmode\hypertarget{ref-stat}{}%
Department of Health. 2021b. \emph{Latest vaccine statistics}. Republic
of South Africa. {[}Online{]}, Available:
\url{https://sacoronavirus.co.za/latest-vaccine-statistics/}.

\leavevmode\hypertarget{ref-herd}{}%
Department of Health. 2021a. \emph{COVID-19 coronavirus vaccine}.
Republic of South Africa. {[}Online{]}, Available:
\url{https://www.gov.za/covid-19/vaccine/vaccine}.

\leavevmode\hypertarget{ref-report}{}%
Department of Health. 2021c. \emph{South africa COVID-19 and vaccine
social listening report}. Republic of South Africa. {[}Online{]},
Available:
\url{https://sacoronavirus.b-cdn.net/wp-content/uploads/2021/10/SA-social-listening-report-5-Oct-2021.pdf}.

\leavevmode\hypertarget{ref-random}{}%
Duflo, E., Glennerster, R. \& Kremer, M. 2007. Using randomization in
development economics research: A toolkit. \emph{Handbook of development
economics}. 4:3895--3962.

\leavevmode\hypertarget{ref-immun}{}%
Fontanet, A. \& Cauchemez, S. 2020. COVID-19 herd immunity: Where are
we? \emph{Nature reviews. Immunology}. 20:583--584.

\leavevmode\hypertarget{ref-duck}{}%
Gandhi, L., Milkman, K.L., Ellis, S., Graci, H., Gromet, D., Mobarak,
R., Buttenheim, A., Duckworth, A., et al. 2021. An experiment evaluating
the impact of large-scale, high-payoff vaccine regret lotteries.
{[}Online{]}, Available: \url{http://dx.doi.org/10.2139/ssrn.3904365}.

\leavevmode\hypertarget{ref-adhere}{}%
Humphrey, C.H., Small, D.S., Jensen, S.T., Volpp, K.G., Asch, D.A., Zhu,
J. \& Troxel, A.B. 2019. Modeling lottery incentives for daily
adherence. \emph{Statistics in Medicine}. 38(15):2847--2867.

\leavevmode\hypertarget{ref-regr}{}%
Husain, S.A., Diaz, K.M., Schwartz, J.E., Parsons, F.E., Burg, M.M.,
Davidson, K.W. \& Kronish, I.M. 2019. Behavioral economics
implementation: Regret lottery improves mHealth patient study adherence.
\emph{Contemporary Clinical Trials Communications}. 15:100387.

\leavevmode\hypertarget{ref-nids}{}%
Ingle, B., K. 2021. \emph{National income dynamics study -- coronavirus
rapid mobile survey (NIDS-CRAM) 2020 - 2021 panel user manual.} Cape
Town: Southern Africa Labour; Development Research Unit: Bureau for
Economic Research.

\leavevmode\hypertarget{ref-bad}{}%
Kadkhoda, K. 2021. Herd immunity to COVID-19: Alluring and elusive.
\emph{American Journal of Clinical Pathology}. 155(4):471--472.

\leavevmode\hypertarget{ref-fast}{}%
Kahneman, D. 2011. \emph{Thinking, fast and slow}. Macmillan.

\leavevmode\hypertarget{ref-prospect}{}%
Kahneman, D. \& Tversky, A. 1979. Prospect theory: An analysis of
decision under risk. \emph{Econometrica}. 47(2):263--291.

\leavevmode\hypertarget{ref-Texevier}{}%
Katzke, N.F. 2017. \emph{{Texevier}: {P}ackage to create elsevier
templates for rmarkdown}. Stellenbosch, South Africa: Bureau for
Economic Research.

\leavevmode\hypertarget{ref-health}{}%
Loewenstein, G., Asch, D., Friedman, J., Melichar, L. \& Volpp, K. 2012.
Can behavioural economics make us healthier? \emph{BMJ (Clinical
research ed.)}. 344:e3482.

\leavevmode\hypertarget{ref-flu}{}%
Madrian, B. 2014. Applying insights from behavioral economics to policy
design. {[}Online{]}, Available: \url{http://dx.doi.org/10.3386/w20318}.

\leavevmode\hypertarget{ref-gamble}{}%
National Gambling Board. 2020. \emph{National gambling statistics:
Casinos, bingo, limited pay-out machines and betting on horse racing and
sport offered by bookmakers and totalisators. Audited statistics:2019/20
financial year}. {[}Online{]}, Available:
\url{https://www.ngb.org.za/SiteResources/documents/2020-21/Stats\%20Presentation\%20FY2019-20\%20pdf.pdf}.

\leavevmode\hypertarget{ref-lotto}{}%
National Lotteries Commission. 2019. \emph{2019 integrated report}.
{[}Online{]}, Available:
\url{http://www.nlcsa.org.za/ir/ir2019/NLC_ir2019/downloads/nlc-full.pdf\#view=Fit}.

\leavevmode\hypertarget{ref-rabin}{}%
Rabin, M. 2002. A perspective on psychology and economics.
\emph{European economic review}. 46(4-5):657--685.

\leavevmode\hypertarget{ref-evds}{}%
Republic of South Africa. 2021. \emph{Electronic vaccination data system
(EVDS) self registration portal}. {[}Online{]}, Available:
\url{https://www.gov.za/covid-19/vaccine/evds}.

\leavevmode\hypertarget{ref-sah}{}%
South African Health Products Regulatory Authority. 2021. \emph{MEDIA
RELEASE ON THE APPROVAL PROCESS OF COVID VACCINES}. {[}Online{]},
Available:
\url{https://www.sahpra.org.za/wp-content/uploads/2021/07/SAHPRA-CovidVaccineApproval_Media-Release_1July2021_FINAL.pdf}.

\leavevmode\hypertarget{ref-statsa}{}%
Statistics South Africa. 2020. Gross domestic product fourth quarter
2019: STATISTICAL RELEASE P0441. {[}Online{]}, Available:
\url{https://www.statssa.gov.za/publications/P0441/P04414thQuarter2019.pdf}.

\leavevmode\hypertarget{ref-nudge}{}%
Thaler, R. \& Sunstein, C. 2008. \emph{Nudge: Improving decisions about
health, wealth, and happiness.} New Haven, CT: Yale University Press.

\leavevmode\hypertarget{ref-khan}{}%
Tversky, A. \& Kahneman, D. 1974. Judgment under uncertainty: Heuristics
and biases. \emph{Science}. 185(4157):1124--1131.

\leavevmode\hypertarget{ref-ref}{}%
Tversky, A. \& Kahneman, D. 1991. Loss aversion in riskless choice: A
reference-dependent model. \emph{The Quarterly Journal of Economics}.
106(4):1039--1061. {[}Online{]}, Available:
\url{http://www.jstor.org/stable/2937956}.

\leavevmode\hypertarget{ref-ohio}{}%
Walkey, A.J., Law, A. \& Bosch, N.A. 2021. Lottery-based incentive in
ohio and COVID-19 vaccination rates. \emph{JAMA}. 326(8):766--767.

\leavevmode\hypertarget{ref-bank}{}%
World Bank. 2020. \emph{The global economic outlook during the COVID-19
pandemic: A changed world}. {[}Online{]}, Available:
\url{https://www.worldbank.org/en/news/feature/2020/06/08/the-global-economic-outlook-during-the-covid-19-pandemic-a-changed-world}.

\leavevmode\hypertarget{ref-who}{}%
World Health Organisation. 2021. \emph{WHO coronavirus (COVID-19)
dashboard}. {[}Online{]}, Available:
\url{https://covid19.who.int/table}.

\end{CSLReferences}

\bibliography{Tex/ref}





\end{document}
